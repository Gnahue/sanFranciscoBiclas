\section{METODOLOGÍA}

Como lo demuestra el "No Free Luch Theorem" no existe un mejor algoritmo para cada categoría de problemas en Machine Learning.  Por eso es que vamos a estudiar varios y decidir cuál, o la combinación de cuáles nos brinda un mejor modelo para predecir un nuevo evento siguiendo el set de datos de la competencia.
Comenzaremos con algoritmos simples y de a poco intentremos refinarlos.

\subsection{Sckit-Learn}

En nuestra búsqueda de información y recomendaciones para el desarrollo de nuestro modelo encontramos Sckit learn, una librería para Python. 
En la documentación figura una diagrama que nos guía para encontrar posibles estimadores para nuestro problema.

\begin{figure}[h]
\centering
\includegraphics[height=10cm]{imagenes/sckit.png}
\label{fig:exemplo}
\end{figure}

\subsection{SGD Regresor}
Dado a que tenemos una cantidad de datos mucho mayor a 100k probaremos con SGD Regresor


\begin{table}[h]
\centering
\caption{ Modelo de como as tabelas devem ser inseridas no texto }
\vspace{0.2in}
\newcolumntype{C}{>{\centering\arraybackslash}X}%
\newcommand{\rowstyle}[1]{%
  \protected\gdef\currentrowstyle{#1}%
}
\begin{tabularx}{\textwidth}{>{\bf}C|C|C|C}
\hline 
\textbf {Índice} & \textbf{Coluna 01} &\textbf{ Coluna 02} & \textbf{Coluna 03} \\ \hline \hline
Linha 01 & & & \\ \hline
Linha 02 & & & \\ \hline                         

\end{tabularx}
\end{table}






\begin{table}[h]
\centering
\caption{ Modelo de como as tabelas devem ser inseridas no texto }
\vspace{0.2in}
\newcolumntype{C}{>{\centering\arraybackslash}X}%
\newcommand{\rowstyle}[1]{%
  \protected\gdef\currentrowstyle{#1}%
}
\begin{tabularx}{\textwidth}{>{\bf}C|C|C|C}
\hline 
\textbf {Índice} & \textbf{Coluna 01} &\textbf{ Coluna 02} & \textbf{Coluna 03} \\ \hline \hline
Linha 01 & & & \\ \hline
Linha 02 & & & \\ \hline                         

\end{tabularx}
\end{table}