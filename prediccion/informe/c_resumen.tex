\section*{\centering{RESUMEN}}
Para poder llevar a cabo este trabajo debimos entender el objetivo de la competencia y las condiciones a respetar. Predecir la duración de viajes en bicicleta realizados en la Bahía de San Francisco.
Se realizó un análisis exploratorio de los datos disponibles, se los visualizaron y se obtuvieron conclusiones. Con esta información procedimos a la preparación de los mismos, limpieza, selección y transformación.
En simultáneo se estudiaron algoritmos de machine learning útiles para casos de regresión. Decidimos orientarnos a árboles de decisión, planteamos el uso de Random Forest y Bagging, ambos combinaciones de árboles predictores. 
Una vez implementados estos algoritmos procedimos a la etapa de perfeccionamiento, prueba y ajuste de hiper-parámetros. Logrando así un mejoramiento de las predicciones.
Por último ensamblamos los resultados obtenidos a través de ambas técnicas y se cargaron en la competencia.